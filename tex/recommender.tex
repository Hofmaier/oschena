\documentclass[a4paper, 11pt]{article}
\author{Lukas Hofmaier}
\title{Recommender}
\begin{document}
\maketitle
\section{User-user collaborative filtering}
\subsection{Similarity}
Um zwei User miteinander zu vergleichen, kann die Pearson Similarity eingesetzt werden.
Was passiert mit User die keine gemeinschaftlichen Item gerated haben? 
Wenn es keinen User gibt, der das selbe Item bewertet hat, ist der Korrelationskoeffizient -1.
Die Pearson Similarity beruecksichtigt nur gemeinsam geratete Daten.
Was passiert wenn zwei User ein einziges gemeinsames Item haben und dieses gleich bewerten?


\begin{equation}
 sim(u,v) = \frac{(r_{u,i} - \bar{r}_u)(r_{v,i} - \bar{r}_v)}{r_{v,i} - \bar{r}_v}    
\end{equation}
Dann zeigt der Pearson Correlation Coffezient eine hohe Aehnlichkeit an.
\end{document}
